\section{Considerazioni finali e valutazione}\label{conclusioni}

TicketOne nel complesso è una piattaforma fatta seguendo le regole di progettazione del web e tenendo conto dei bisogni dell'utente e di come questo si muove all'interno del sito.
\par Di seguito la tabella con le valutazioni sui vari temi analizzati:
\begin{center}
	\renewcommand{\arraystretch}{2}
	\setlength{\tabcolsep}{1.3cm} % for the horizontal padding
    \begin{tabular}{| c | c |}
		\hline
        \textbf{Descrizione} & \textbf{Valutazione} \\
        \hline \hline
        Struttura generale del sito & 8 \\\hline 
        Who? & 9 \\\hline 
        What? & 9 \\\hline 
        When? & 8 \\\hline 
        How? & 8 \\\hline
        Where? & 7 \\\hline 
        Paragrafo, contenuti e link & 8 \\\hline
        Pubblicità & 10 \\\hline 
        Ricerca & 8 \\\hline
        Mobile & 8 \\\hline \hline
        \textbf{Complessivo} & \textbf{8} \\\hline
    \end{tabular}
	\renewcommand{\arraystretch}{1}% for the vertical padding
\end{center}
I voti dati a ciascuna sezione vanno da 1 a 10 dove:
\begin{itemize}[noitemsep]
	\item 1 è inutilizzatile / pessimo;
	\item 6 è a malapena utilizzabile / sufficiente;
	\item 10 è perfettamente utilizzabile / ottimo.
\end{itemize}
Nel complesso questo sito è stato concepito, sviluppato e mantenuto tenendo sempre a mente la sua \textit{mission}: di permettere all'utente di trovare e comprare i biglietti per gli eventi ai quali vuole assistere, siano essi di musica, cultura o sportivi.
\par La sezione dell'asse Where ha preso una valutazione di 7 in quanto TicketOnet non utilizza particolari metodi per dire all'utente la sua posizione all'interno del sito oppure indicando le pagine che ha visitato in precedenza.
Questo voto sarebbe stato maggiore se fosse presente un'indicazione che dice dove l'utente si trova.
\par Il voto 10 della sezione Pubblicità rispecchia l'ottimo posizionamento di questa all'interno del sito tramite l'uso del blending.
\par Nella sezione della ricerca il voto 8 rappresenta non solamente la versione mobile del sito, adattata dal desktop, ma anche la possibilità di scaricare l'applicazione che fornisce una miglior interfaccia utente.
\par TicketOne, nonostante i voti così alti nella tabella, potrebbe essere migliorato nella parte di presentazione dei risultati di ricerca che, nonostante facciano trovare facilmente gli eventi agli utenti, sono presentati in una lista senza troppe opzioni.
