\section{Sommario}\label{sommario}

\par Internet è parte integrante nella giornata di tutti, non solamente con lo scopo di intrattenere ma anche per motivi lavorativi, rendendo \textit{essenziale} il modo in cui vengono organizzate le informazioni per poterle trovare al meglio e navigare facilmente nella rete.
\'E per questo che i siti, paragonabili a contenitori di informazioni, vanno pensati e realizzati in maniera da essere il più fruibili possibile, indifferentemente che siano destinati ad uso commerciale o meno. 
\par Ormai è veramente facile realizzare un sito web grazie alle moltissime soluzioni ``prefabbicate'' che permettono l'utilizzo di \textit{template} in cui inserire solamente testo e immagini, modificando alcune impostazioni quali colori e font (se non si sa utilizzare i linguaggi usati nella rete).
L'uso di questi template non è sempre adatto a tutti i tipi di siti, anche con scopi differenti, e se adattati poco oppure male possono risultare errori quali \textit{bloated design} o una cattiva organizzazione delle informazioni, rendendo quindi il sito impossibile da navigare.
\par Usare soluzioni che permettono di fare un sito a partire da un template non significa che questo risulti facilmente fruibile dagli utenti finali e non ne garantisce un buon posizionamento nelle classifiche dei motori di ricerca.
\'E per questo che per progetti più ampi ci si rivolge ad agenzie web che mettono a disposizione non solamente la figura dello sviluppatore ma anche quella di graphic designer, SEO engineer, marketing, ecc.

\bigskip In questa relazione viene effettuata un'analisi dell'usabilità del sito \url{https://www.ticketone.it/}.
Il periodo relativo a questa è la seconda metà di Maggio 2020.

\par \textit{TicketOne} è una piattaforma online, nata nel 1998, che offre servizi di biglietteria, marketing, informazione e commercio elettronico per eventi di musica, spettacolo, sport e cultura a livello internazionale.
Maggiori informazioni riguardo l'azienda che detiene TicketOne possono essere trovate nella loro pagina ``\textit{\href{https://www.ticketone.it/campaign/chisiamo/}{Chi siamo}}''.

\begin{figure}[hbt]
	\centering
	\includegraphics[width=7cm]{img/ticketone_logo.jpeg}
	\caption{Logo \textit{TicketOne}}
	\label{fig:logo}
\end{figure}

\newpage
\subsection{Organizzazione del documento}
	\par L'elaborato è organizzato come segue:
	\begin{enumerate}
		\item Sez.~\ref{sommario} \textit{Sommario}: introduzione al documento e ai suoi contenuti;
		\item Sez.~\ref{analisipre} \textit{Analisi preliminare}: analisi del homepage e della struttura;  
		\item Sez.~\ref{6w} \textit{Le 6W}: analisi di come vengono applicate le \textit{6 W} nel sito;
		\item Sez.~\ref{contenav} \textit{Contenuto e navigazione}: analisi su contenuto e navigazione; 
		\item Sez.~\ref{pubblicita} \textit{Pubblicità}: come la pubblicità viene integrata nel sito; 	
		\item Sez.~\ref{ricerca} \textit{Ricerca}: come vengono implementate le funzioni di ricerca;
		\item Sez.~\ref{mobile} \textit{Versione mobile}: analisi della versione mobile del sito;
		\item Sez.~\ref{conclusioni} \textit{Considerazioni finali e valutazione}: conclusioni e valutazioni. 
	\end{enumerate}

\subsection{Allegati}\label{allegati}

	Insieme a questa relazione sono presenti i seguenti allegati, che rappresentano le pagine principali che sono state considerate all'interno di questa relazione:
	\begin{itemize}
		\item \textit{allegato\_1\_homepage.pdf}: screenshot per intero del homepage del sito;
		\item \textit{allegato\_2\_ricerca\_notre\_dame\_de\_paris.pdf}: screenshot per intero dei risultati di ricerca di ``\textit{Notre Dame de Paris}'';
		\item \textit{allegato\_3\_evento\_notre\_dame\_de\_paris.pdf}: screenshot per intero di una pagina di evento, in particolare ``\textit{Notre Dame de Paris}'';
		\item \textit{allegato\_4\_ricerca\_eventi\_padova.pdf}: screenshot per intero di una pagina con gli eventi a Padova;
		\item \textit{allegato\_5\_evento\_willie\_peyote.pdf}: screenshot per intero di una pagina di evento, in particolare un concerto di ``\textit{Willie Peyote}''.
	\end{itemize}
	
